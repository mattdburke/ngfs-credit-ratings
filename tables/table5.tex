
\begin{table}[tb!]
\footnotesize
\centering
\caption{S\&P Credit Rating Scale}
\label{tab:sp_rating_scale}
\begin{tabularx}{\textwidth}{p{4cm} p{3cm} p{4cm} }
\hline
Long-term Foreign Currency Issuer Rating Symbol & Numerical Rating & Grade \\
\hline
AAA & 20 & Prime \\
AA+ & 19 & High \\
AA & 18 & \\
AA- & 17 & \\
A+ & 16 & Upper-medium \\
A & 15 & \\
A- & 14 & \\
BBB+ & 13 & Lower-medium \\
BBB & 12 & \\
BBB- & 11 & \\
BB+ & 10 & Speculative \\
BB & 9 & \\
BB- & 8 & \\
B+ & 7 & Highly speculative \\
B & 6 & \\
B- & 5 & \\
CCC+ & 4 & Substantial risks \\
CCC & 3 & \\
CCC- & 2 & \\
CC & 1 & Extremely speculative / In default \\
C & 1 & \\
D/SD & 1 & \\
\hline
\multicolumn{3}{p{\textwidth}}{\begin{footnotesize}This table shows the S\&P sovereign credit rating scale. In this study, when we make reference to notch changes in a rating, we refer to this scale.
\end{footnotesize}
}
\end{tabularx}
\end{table}
